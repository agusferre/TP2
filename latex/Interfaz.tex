\-La interfaz del diccionario es muy parecida a la usada en la biblioteca estándar en su versión \-C++03. \-El diccionario se implementa usando una clase llamada \hyperlink{classaed2_1_1map}{aed2\-::map}. \-Vale notar el uso del namespace \hyperlink{namespaceaed2}{aed2} para evitar conflictos con std\-::map.

\-La clase \hyperlink{classaed2_1_1map}{aed2\-::map} es un {\itshape template\/} (clase paramétrica) cuyos parámetros de tipo son\-:
\begin{DoxyItemize}
\item \-Key (aed2\-::map\-::key\-\_\-type)\-: tipo de las claves.
\item \-Meaning (\#aed2\-::map\-::mapped\-\_\-type)\-: tipo de los significados.
\item \-Compare (\#aed2\-::map\-::key\-\_\-compare)\-: functor de comparación. \-Puede ser cualquier objeto que se pueda invocar con dos parámetros de tipo \-Key, retornando {\ttfamily bool}. \-La función $<$ así definida debe ser un orden total. \-Notar que dentro de la implementación se considera que {\ttfamily k} y {\ttfamily k'} representan la misma clave si y sólo si tanto {\ttfamily k} $<$ {\ttfamily k'} como {\ttfamily k'} $<$ {\ttfamily k} retornan {\ttfamily false}.
\end{DoxyItemize}\hypertarget{Interfaz_Diferencias}{}\subsection{\-Significados vs. Valores}\label{Interfaz_Diferencias}
\-En la materia, el uso del diccionario se presenta diferenciando claramente los roles de las claves y los significados. \-Esta diferenciación hace más simple y abstracto el manejo del diccionario. \-Por diversas razones, entre las que se incluyen la compatibilidad con la biblioteca de algoritmos, \-C++ toma una postura diferente. \-En lugar de explicitar que un diccionario es una asociación de claves a significados, en \-C++ un diccionario es una colección de pares `(k, s)`. \-Obviamente, el diccionario incluye las funciones requeridas para manipular el diccionario a través de las claves, pero no separa los valores de los significados. \-Entender esta diferencia es importante para no llevarse algunas sorpresas. \-Por ejemplo, en \-A\-E\-D2 solemos definir un significado `s` a una clave `k` de un diccionario `d` invocando\-: 
\begin{DoxyCode}
 \{.unparsed\}
 d.definir(k, s).
\end{DoxyCode}
 \-En cambio, en \-C++, se {\itshape inserta un par\/} `(k, s)` en el diccionario `d`. \-Si disponemos de `k` y `s` por separado, vamos a tener que generar el par antes de insertar\-: 
\begin{DoxyCode}
 \{.cpp\}
 d.insert(std::make\_pair(k, s))
\end{DoxyCode}


\-Como segundo ejemplo, en la materia uno suele verificar si una clave esta definida para acceder a su significado, invocando algo como\-: 
\begin{DoxyCode}
 \{.unparsed\}
 \textcolor{keywordflow}{if}(d.definido(k)) \{ ... Meaning s = d.significado(k) ... \}
\end{DoxyCode}
 \-Obviamente, podemos evitar la doble búsqueda si `significado` retorna un handle (i.\-e., puntero) al significado\-: 
\begin{DoxyCode}
 \{.unparsed\}
 Meaning s = d.significado(k)
 \textcolor{keywordflow}{if}(s != d.null()) \{ *s += 1; \}
\end{DoxyCode}
 \-En cualquier caso, lo importante es que `s` es un significado. \-En \-C++, en cambio, la función de búsqueda (llamada `find`) retorna el par `(k, s)` (obviamente, `k` no es modificable), con lo cual el código anterior se transforma en\-: 
\begin{DoxyCode}
 \{.cpp\}
 \textcolor{keyword}{auto} v = d.find(k)
 \textcolor{keywordflow}{if}(v != d.end()) \{
    assert(k == v->first); \textcolor{comment}{//la clave es el primero!}
    v->second += 1;        \textcolor{comment}{//el significado es el segundo!}
 \}
\end{DoxyCode}


\-En el ejemplo anterior, usamos `auto` para definir el tipo de `v`. \-Los tipos correctos de `v` y `$\ast$v` son\-: 
\begin{DoxyCode}
 \{.cpp\}
 std::map<const Key, Meaning>::iterator v
 std::pair<const Key, Meaning>& *v
\end{DoxyCode}
 \-Este ejemplo muestra la importancia de tener definidos `key\-\_\-type` y `mapped\-\_\-type`, ya que si no podemos acceder a \-Key y \-Meaning pero sí a un renombre \-T del tipo std\-::map (e.\-g., en una funcion template), aún podemos acceder al tipo del valor apuntado por el iterador escribiendo\-: 
\begin{DoxyCode}
 \{.cpp\}
 std::map<const typename T::key\_type, typename T::meaning>&
\end{DoxyCode}
 para poder describir el tipo. \-Igualmente, es muy incomodo escribir el tipo así y es por eso que la clase también incluye el siguiente renombre\-: 
\begin{DoxyCode}
 \{.cpp\}
 \textcolor{keyword}{using} value\_type = std::pair<const key\_type, mapped\_type>;
\end{DoxyCode}
 \-De esta forma, podemos describir el tipo anterior como `typename \-T\-::value\-\_\-type`.

\-La clase viene equipada con otros renombres para respetar el estándar, pero los vamos a ignorar por ahora. (\-Nota\-: recordar los traits de la clase de templates)\hypertarget{Interfaz_Iteradores}{}\subsection{\-Iteradores}\label{Interfaz_Iteradores}
\-El módulo que implementamos viene equipado con iteradores que siguen el estándar \-C++. \-Estos iteradores son bidireccionales, ya que también permiten recorrer la estructura en el orden inverso. \-Obviamente, como estamos implementando un diccionario {\bfseries ordenado}, la iteración sigue el orden definido por las claves de acuerdo al functor de comparación (aed2\-::map\-::key\-\_\-compare).

\-Como vemos en la materia, los iteradores no se usan exclusivamente para recorrer la estructura, sino que también se usan como {\itshape handles\/} (punteros) a la misma. \-De esta forma, podemos acceder a los valores en $O$(1) sin exhibir la implementación. \-La desventaja de este esquema (que queda opacada en cuanto a sus ventajas) es que los iteradores se pueden invalidar cuando la estructura cambia. \-En esta implementación, salvo excepciones puntuales bien documentadas, se {\bfseries garantiza} que un iterador se invalida únicamente cuando se elimina el valor referenciado por el mismo. \-Ninguna otra operación puede invalidar el iterador (obviamente, la secuencia iterada sí cambia, pero no se invalida).

\-Aprovechando esta interpretación de iteradores como handles, la interfaz de \-C++ tiene muchas funciones que retornan iteradores, y algunas que se aprovechan de la existencia de los iteradores para mejorar el tiempo de computo. \-Por ejemplo, es posible eliminar un valor en $O$(1) (amortizado) si se provee un iterador apuntando al valor a eliminar \cite{MehlhornSanders2008}.

\-Con respecto a la eficiencia, vamos a {\bfseries garantizar} que los iteradores son {\itshape livianos\/} (i.\-e., ocupan $O$(1) memoria y son copiables en $O$(1) tiempo). \-En consecuencia, se pueden pasar iteradores por copia sin ningún costo. \-Asimismo, vamos a {\bfseries garantizar} que la creación de iteradores apuntando a la primer posición (`begin`) y la posición pasando-\/el-\/fin (`end`) toman tiempo $O$(1). \-Esto nos permite verificar si llegamos al fin del recorrido en $O$(1) tiempo, de la siguiente forma\-: 
\begin{DoxyCode}
 \{.cpp\}
 \textcolor{keywordflow}{for}(\textcolor{keyword}{auto} it = d.begin(); it != d.end(); ++it) \{...\}   \textcolor{comment}{//cada comparacion en
       O(1)}
 \textcolor{keywordflow}{for}(\textcolor{keyword}{auto} it = d.rbegin(); it != d.rend(); ++it) \{...\} \textcolor{comment}{//cada comparacion en
       O(1)}
\end{DoxyCode}
 \-Asimismo, podemos determinar si una clave existe, comparando el resultado de `find` con el iterador `d.end()`, usando $O$(1) tiempo para la comparación.

\begin{DoxySeeAlso}{\-Ver también}
\mbox{[}\-Documentación de std\-::map\mbox{]}(\href{http://en.cppreference.com/w/cpp/container/map}{\tt http\-://en.\-cppreference.\-com/w/cpp/container/map})\par
 \mbox{[}\-Documentación de \-Bidirectional\-Iterator\mbox{]}(\href{http://en.cppreference.com/w/cpp/concept/BidirectionalIterator}{\tt http\-://en.\-cppreference.\-com/w/cpp/concept/\-Bidirectional\-Iterator}) 
\end{DoxySeeAlso}
