\-Namespace para las clases de \-A\-E\-D2.

\-En esta pagina, y por conveniencia, se listan todos los axiomas y proposiciones auxiliares requeridos para formalizar los invariantes de representación y las funciones de abstracción. \-Previamente se presentan los renombres de los tipos usados.

\begin{DoxyParagraph}{\-Renombres de tipos}

\end{DoxyParagraph}

\begin{DoxyItemize}
\item \-Node es tupla(child\-: arreglo\mbox{[}2\mbox{]} de puntero(\-Node), parent\-: puntero(\-Node), color\-: \-Color, value\-: \-Value)
\item \-Value es \-Maybe(value\-\_\-type)
\item value\-\_\-type es tupla(clave\-: \-Key, significado\-: \-Meaning)
\end{DoxyItemize}

\-El \-T\-A\-D \-Maybe( $\alpha$) representa un tipo $\alpha$ extendido con un valor nulo. \-En otras palabras, el \-T\-A\-D \-Maybe se puede usar para representar los valores de los nodos, donde el nodo cabecera no tiene valor y los nodos internos sí tienen valor. (\-Esto independientemente de si se implementa con herencia o con un puntero o de otra forma.) \-Tiene dos observadores\-:
\begin{DoxyItemize}
\item nothing?(x)\-: que indica si x tiene un valor nulo, y
\item data(x)\-: que devuelve el valor de x, suponiendo que no es inválido.
\end{DoxyItemize}

\-La especificación de este \-T\-A\-D queda como ejercicio (no obligatorio).\hypertarget{axiomas_sec-Axiomas}{}\subsection{\-Axiomas y proposiciones auxiliares}\label{axiomas_sec-Axiomas}
\-En esta sección se deben incluir todos los axiomas y proposiciones auxiliares que se usen para describir los invariantes de representación, las funciones de abstracción, las precondiciones y las postcondiciones.

\begin{DoxyRemark}{\-Comentarios}
\-Recordar incluir un alias en el archivo doxyfile a fin de poder referenciar automaticamente a cada axioma desde las otras páginas.
\end{DoxyRemark}
\-Se muestran algunos ejemplos a continuación.

\begin{DoxyParagraph}{es\-Diccionario?}
\-Retorna true si la secuencia representa un diccionario
\end{DoxyParagraph}
es\-Diccionario?\-: secu(tupla( $\alpha$, $\beta$)) $\to$ bool\par
 es\-Diccionario?(s) $\equiv$ sin\-Repetidos?(\href{axiomas.html#primeros}{\tt primeros}(s)) 

\begin{DoxyParagraph}{primeros}
\-Proyecta las primeras componentes de una secuencia de pares
\end{DoxyParagraph}
primeros\-: secu(tupla( $\alpha$, $\beta$)) $\to$ secu( $\alpha$)\par
 primeros(s) $\equiv$ {\bfseries if} vacia?(s) {\bfseries then} $<$$>$ {\bfseries else} $\pi_1$(prim(s)) $\bullet$ \href{axiomas.html#primeros}{\tt primeros}(fin(s)) {\bfseries fi} 

\begin{DoxyParagraph}{\-Hasta\-Elem}
\-Devuelve una secuencia ordenada con elementos del diccionario hasta el elemento e, no inclusive. \-Hasta\-Elem\-: \-Conj( $\alpha$, $\beta$) x $\to$ secu( $\alpha$, $\beta$)\par
 \-Hasta\-Elem(claves, e) $\equiv$ {\bfseries if} minimo(d) = e {\bfseries then} $<$$>$ {\bfseries else} \-Hasta\-Elem(claves -\/ \{minimo\}, e) 
\end{DoxyParagraph}
\begin{DoxyParagraph}{\-Desde\-Elem}
\-Devuelve una secuencia ordenada con elementos del diccionario desde el elemento e inclusive. \-Desde\-Elem\-: \-Conj( $\alpha$) x $\to$ secu( $\alpha$, $\beta$)\par
 \-Desde\-Elem(claves, e) $\equiv$ {\bfseries if} minimo(d) $<$ e {\bfseries then} $<$$>$ {\bfseries else} minimo(d) $\bullet$ \-Desde\-Elem(claves -\/ \{minimo\}, e) 
\end{DoxyParagraph}
\begin{DoxyParagraph}{\-Menor\-Que}
\-Devuelve true si y solo si m1 es lexicograficamente menor que m2. \-Menor\-Que\-: \-Dicc( $\alpha$, $\beta$) d1 x \-Dicc( $\alpha$, $\beta$) d2 $\to$ bool\par
 \-Menor\-Que(d1, d2) $\equiv$ comparar\-Claves(claves(d1), claves(d2)) 
\end{DoxyParagraph}
\begin{DoxyParagraph}{comparar\-Claves}
comparar\-Claves\-: \-Conj() cs1 x \-Conj() cs2 $\to$ bool\par
 comparar\-Claves(cs1, cs2) $\equiv$ {\bfseries if} \# cs1 == 0 {\bfseries then} true {\bfseries else} {\bfseries if} (\# cs2 == 0 $\lor$ minimo(cs1) $>$ minimo(cs2) ) {\bfseries then} false {\bfseries fi} \-E\-L\-S\-E comparar\-Claves(cs1 -\/ \{minimo\}, cs2 -\/ \{minimo\}) {\bfseries fi} 
\end{DoxyParagraph}
\begin{DoxyParagraph}{minimo}
\-Devuelve el elemento del diccionario con la menor clave minimo\-: \-Conj( $\alpha$, $\beta$) $\to$ ( $\alpha$, $\beta$)\par
 minimo(claves) $\equiv$ minimo\-Aux(claves, dame\-Uno(claves)) 
\end{DoxyParagraph}
\begin{DoxyParagraph}{minimo\-Aux}
mínimo\-Aux\-: \-Conj( $\alpha$, $\beta$) x ( $\alpha$, $\beta$) $\to$ ( $\alpha$, $\beta$)\par
 minimo\-Aux(claves, e) $\equiv$ {\bfseries if} \# claves = 0 {\bfseries then} e {\bfseries else} minimo\-Aux(sin\-Uno(claves), max(e, dame\-Uno(claves))) {\bfseries fi}  \#include $<$functional$>$ \#include $<$iterator$>$ \#include $<$utility$>$ \#include $<$cassert$>$ \#include $<$algorithm$>$ \#include $<$iostream$>$
\end{DoxyParagraph}
\#ifdef \-D\-E\-B\-U\-G \-Aca se puede incluir cualquier cosa que consideren que necesitan para debug \#include $<$iostream$>$ \#include $<$iomanip$>$ \#endif

/$\ast$$\ast$ \-Para evitar cualquier tipo de conflictos, todas las clases de \-A\-E\-D2 se implementan en el namespace aed2. \-Recordar el uso de `namespace`, el keyword `using`, y el método \-A\-D\-L para la busqueda de las funciones. 