\-En esta pagina, y por conveniencia, se listan todos los axiomas y proposiciones auxiliares requeridos para formalizar los invariantes de representación y las funciones de abstracción. \-Previamente se presentan los renombres de los tipos usados.

\begin{DoxyParagraph}{\-Renombres de tipos}

\end{DoxyParagraph}

\begin{DoxyItemize}
\item \-Node es tupla(child\-: arreglo\mbox{[}2\mbox{]} de puntero(\-Node), parent\-: puntero(\-Node), color\-: \-Color, value\-: \-Value)
\item \-Value es \-Maybe(value\-\_\-type)
\item value\-\_\-type es tupla(clave\-: \-Key, significado\-: \-Meaning)
\end{DoxyItemize}

\-El \-T\-A\-D \-Maybe( $\alpha$) representa un tipo $\alpha$ extendido con un valor nulo. \-En otras palabras, el \-T\-A\-D \-Maybe se puede usar para representar los valores de los nodos, donde el nodo cabecera no tiene valor y los nodos internos sí tienen valor. (\-Esto independientemente de si se implementa con herencia o con un puntero o de otra forma.) \-Tiene dos observadores\-:
\begin{DoxyItemize}
\item nothing?(x)\-: que indica si x tiene un valor nulo, y
\item data(x)\-: que devuelve el valor de x, suponiendo que no es inválido.
\end{DoxyItemize}

\-La especificación de este \-T\-A\-D queda como ejercicio (no obligatorio).\hypertarget{axiomas_sec-Axiomas}{}\subsection{\-Axiomas y proposiciones auxiliares}\label{axiomas_sec-Axiomas}
\-En esta sección se deben incluir todos los axiomas y proposiciones auxiliares que se usen para describir los invariantes de representación, las funciones de abstracción, las precondiciones y las postcondiciones.

\begin{DoxyRemark}{\-Comentarios}
\-Recordar incluir un alias en el archivo doxyfile a fin de poder referenciar automaticamente a cada axioma desde las otras páginas.
\end{DoxyRemark}
\-Se muestran algunos ejemplos a continuación.

\begin{DoxyParagraph}{es\-Diccionario?}
\-Retorna true si la secuencia representa un diccionario
\end{DoxyParagraph}
es\-Diccionario?\-: secu(tupla( $\alpha$, $\beta$)) $\to$ bool\par
 es\-Diccionario?(s) $\equiv$ sin\-Repetidos?(\href{axiomas.html#primeros}{\tt primeros}(s)) 

\begin{DoxyParagraph}{primeros}
\-Proyecta las primeras componentes de una secuencia de pares
\end{DoxyParagraph}
primeros\-: secu(tupla( $\alpha$, $\beta$)) $\to$ secu( $\alpha$)\par
 primeros(s) $\equiv$ {\bfseries if} vacia?(s) {\bfseries then} $<$$>$ {\bfseries else} $\pi_1$(prim(s)) $\bullet$ \href{axiomas.html#primeros}{\tt primeros}(fin(s)) {\bfseries fi} 

\begin{DoxyParagraph}{\-Dicc\-A\-Secu}
\-Devuelve una secuencia ordenada con elementos del diccionario.
\end{DoxyParagraph}
\-Dicc\-A\-Secu\-: \-Dicc( $\alpha$, $\beta$) $\to$ secu( $\alpha$, $\beta$)\par
 \-Dicc\-A\-Secu(d) $\equiv$ \-Por\-Clave(claves(d), d) 

\begin{DoxyParagraph}{\-Por\-Clave}
\-Recorre las claves del diccionario para devolver la secuencia ordenada con sus valores.
\end{DoxyParagraph}
\-Por\-Clave\-: \-Conj( $\alpha$) x \-Dicc( $\alpha$, $\beta$) $\to$ secu( $\alpha$, $\beta$)\par
 \-Por\-Clave(cs, d) $\equiv$ {\bfseries if} $\emptyset$ ?(cs) {\bfseries then} $<$ $>$ {\bfseries else} $\langle$ minimo(cs) , obtener(minimo(cs), d) $\rangle$ $\bullet$ \-Por\-Clave(cs -\/ \{minimo(cs)\}, d) {\bfseries fi} 

\begin{DoxyParagraph}{\-Menor\-Que}
\-Devuelve true si y solo si m1 es lexicograficamente menor que m2.
\end{DoxyParagraph}
\-Menor\-Que\-: \-Dicc( $\alpha$, $\beta$) d1 x \-Dicc( $\alpha$, $\beta$) d2 $\to$ bool\par
 \-Menor\-Que(d1, d2) $\equiv$ comparar\-Claves(claves(d1), claves(d2)) 

\begin{DoxyParagraph}{comparar\-Claves}

\end{DoxyParagraph}
comparar\-Claves\-: \-Conj( $\alpha$, $\beta$) cs1 x \-Conj( $\alpha$, $\beta$) cs2 $\to$ bool\par
 comparar\-Claves(cs1, cs2) $\equiv$ {\bfseries if} \# cs1 = 0 {\bfseries then} true {\bfseries else} {\bfseries if} (\# cs2 = 0 $\lor$ $\pi_1$(minimo(cs1)) $>$ $\pi_1$(minimo(cs2)) $|$$|$ (minimo(cs1)) $>$ (minimo(cs2))) {\bfseries then} false {\bfseries fi} {\bfseries else} comparar\-Elems(cs1 -\/ \{minimo\}, cs2 -\/ \{minimo\}) {\bfseries fi} 

\begin{DoxyParagraph}{hasta\-Elem}
\-Devuelve una secuencia de valores hasta el valor con clave = key (no inclusive).
\end{DoxyParagraph}
hasta\-Elem\-: secu( $\alpha$, $\beta$) s x $\alpha$ key $\to$ secu( $\alpha$, $\beta$)\par
 hasta\-Elem(s,k) $\equiv$ {\bfseries if} vacia?(s) $\lor_{\rm L}$ $\pi_1$(prim(s)) = key {\bfseries then} $<$ $>$ {\bfseries else} prim(s) $\bullet$ hasta\-Elem(fin(s), k) {\bfseries fi} 

\begin{DoxyParagraph}{desde\-Elem}
\-Devuelve una secuencia desde el elemento con clave = key inclusive.
\end{DoxyParagraph}
desde\-Elem\-: secu( $\alpha$, $\beta$) s x $\alpha$ key $\to$ secu( $\alpha$, $\beta$)\par
 desde\-Elem(s,k) $\equiv$ {\bfseries if} vacia?(s) {\bfseries then} $<$ $>$ {\bfseries else} {\bfseries if} $\pi_1$(prim(s)) $<$ key {\bfseries then} desde\-Elem(fin(s),k) {\bfseries else} s {\bfseries fi} {\bfseries fi} 

\begin{DoxyParagraph}{header\-Valido}
\-El nodo header no tiene valor. \-Su padre es la raíz, de color negro, y sus hijos derecho e izquierdo son el mayor y el menor valor del arbol respectivamente.
\end{DoxyParagraph}
header\-Valido\-: \-Node n $\to$ bool\par
 header\-Valido(n) $\equiv$ \mbox{[}n.\-color == \-Header $\land$ nothing?(n.\-value) $\land$ n.\-parent.\-color == \-Black $\land$ (n.\-parent = $\bot$ $\land$ (n.\-child\mbox{[}0\mbox{]} == \&header) $\lor$ (n.\-parent $\_linebr eq$ $\bot$ $\Rightarrow_{\rm L}$ n.\-child\mbox{[}0\mbox{]} = llegar\-A\-Minimo(n.\-parent)) $\land$ (n.\-parent = $\bot$ $\land$ (n.\-child\mbox{[}1\mbox{]} == \&header) $\lor$ (n.\-parent $\_linebr eq$ $\bot$ $\Rightarrow_{\rm L}$ n.\-child\mbox{[}1\mbox{]} = llegar\-A\-Maximo(n.\-parent))\mbox{]} 

\begin{DoxyParagraph}{llegar\-A\-Minimo}
\-Devuelve un puntero al minimo nodo del subarbol cuya raiz es el nodo parametro.
\end{DoxyParagraph}
llegar\-A\-Minimo\-: puntero(node) n $\to$ puntero(node) \{n $\_linebr eq$ $\bot$\}\par
 llegar\-A\-Minimo(n) $\equiv$ {\bfseries if} n.\-child\mbox{[}0\mbox{]} == \-N\-U\-L\-L {\bfseries then} n {\bfseries else} llegar\-A\-Minimo(n.\-child\mbox{[}0\mbox{]}) {\bfseries fi} 

\begin{DoxyParagraph}{llegar\-A\-Maximo}
\-Devuelve un puntero al minimo nodo del subarbol cuya raiz es el nodo parametro.
\end{DoxyParagraph}
llegar\-A\-Maximo\-: puntero(node) n $\to$ puntero(node)\par
 llegar\-A\-Minimo(n) $\equiv$ {\bfseries if} n.\-child\mbox{[}1\mbox{]} == \-N\-U\-L\-L {\bfseries then} n {\bfseries else} llegar\-A\-Minimo(n.\-child\mbox{[}1\mbox{]}) {\bfseries fi} 

\begin{DoxyParagraph}{nodos\-Internos\-Validos}
\-Para cada nodo interno, sus ramas descendientes tienen la misma cantidad de nodos negros, y y el subarbol que tiene como raiz ese nodo cumple el invariante de \-Arbol binario de busqueda. \-Si es rojo, su o sus hijos son negros.
\end{DoxyParagraph}
( $\forall$ n\-: \-Node) nodo\-Hijo(n, e.\-header) $\Rightarrow$ (n.\-child\mbox{[}0\mbox{]} = $\bot$ $\lor_{\rm L}$ n.\-child\mbox{[}0\mbox{]}-\/$>$color = \-Black) $\land$ (n.\-child\mbox{[}1\mbox{]} = $\bot$ $\lor_{\rm L}$ n.\-child\mbox{[}1\mbox{]}-\/$>$color = \-Black) $\land$ nodos\-Negros(n.\-child\mbox{[}0\mbox{]}) = nodos\-Negros(n.\-child\mbox{[}1\mbox{]}) $\land$ arbol\-Binario\-De\-Busqueda(n)) 

\begin{DoxyParagraph}{\-Termina}

\end{DoxyParagraph}
\-Termina\-: puntero(node) n x \hyperlink{classNat}{\-Nat} k $\to$ bool\par
 \-Termina(n) $\equiv$ long(inorder(n)) $<$ k 

\begin{DoxyParagraph}{arbol\-Binario\-De\-Busqueda}
\-Todos los nodos de la rama izquierda de n tienen valor menor al de n y todos los de la derecha, mayor.
\end{DoxyParagraph}
arbol\-Binario\-De\-Busqueda\-: puntero(\-Node) n $\to$ bool\par
 arbol\-Binario\-De\-Busqueda(n) $\equiv$ n-\/$>$child\mbox{[}0\mbox{]} $=_{\rm obs}$ \-B\-O\-T\-T\-O\-M $\lor_{\rm L}$ (n-\/$>$child\mbox{[}0\mbox{]}-\/$>$value $<$ n-\/$>$value $\land$ arbol\-Binario\-De\-Busqueda(n-\/$>$child\mbox{[}0\mbox{]}))) $\land$ n-\/$>$child\mbox{[}0\mbox{]} $=_{\rm obs}$ $\bot$ $\lor_{\rm L}$ (n-\/$>$child\mbox{[}0\mbox{]}-\/$>$value $>$ n-\/$>$value $\land$ arbol\-Binario\-De\-Busqueda(n-\/$>$child\mbox{[}0\mbox{]}))) 

\begin{DoxyParagraph}{nodos\-Negros}
\-Devuelve la cantidad de nodos negros en un subarbol.
\end{DoxyParagraph}
nodos\-Negros \-: puntero(\-Node) $\to$ int\par
 nodos\-Negros(n) $\equiv$ {\bfseries if} n $=_{\rm obs}$ $\bot$ $\lor_{\rm L}$ n-\/$>$color = \-Red {\bfseries then} 0 {\bfseries else} 1 {\bfseries fi} + nodos\-Negros(n-\/$>$child\mbox{[}0\mbox{]}) + nodos\-Negros(n-\/$>$child\mbox{[}1\mbox{]}) 

\begin{DoxyParagraph}{nodo\-Hijo}
\-Devuelve true si el n1 es hijo de n2 en la estructura.
\end{DoxyParagraph}
nodo\-Hijo \-: \-Node n1 x \-Node n2 $\to$ bool\par
 nodo\-Hijo(n, h) $\equiv$ $\lnot$ n.\-parent $=_{\rm obs}$ \-B\-O\-T\-T\-O\-M $\land$ ($\ast$n.parent = n2 $\lor$ nodo\-Hijo($\ast$n.parent, n2)) 

\begin{DoxyParagraph}{llegar\-Al\-Header}
\-Dado un puntero a nodo (de un arbol), devuelve un puntero al header.
\end{DoxyParagraph}
\{llegar\-Al\-Header\}\-: puntero(\-Node) $\to$ puntero(\-Node)\{n $\_linebr eq$ $\bot$\}\par
 \href{axiomas.html#llegarAlHeader}{\tt llegar\-Al\-Header}(n) $\equiv$ {\bfseries if} n-\/$>$color = \-Header {\bfseries then} n {\bfseries else} llegar\-Al\-Header(n-\/$>$parent) {\bfseries fi} 

\begin{DoxyParagraph}{inorder}
\-Dado un puntero a un nodo, devuelve una secuencia con los elementos del arbol inorder.
\end{DoxyParagraph}
inorder\-: puntero(\-Node) $\to$ secu( $\alpha$, $\beta$)\par
 inorder(n) $\equiv$ {\bfseries if} n $=_{\rm obs}$ $\bot$ {\bfseries then} $<$ $>$ {\bfseries else} inorder(n-\/$>$child\mbox{[}0\mbox{]}) \& (n-\/$>$value $\bullet$ inorder(n-\/$>$child\mbox{[}1\mbox{]})) {\bfseries fi} 

\begin{DoxyParagraph}{\-Map\-A\-Conjunto}
\-Crea un conjunto con los elementos de map.
\end{DoxyParagraph}
\-Map\-A\-Conjunto\-: \-Node n $\to$ \-Conj( $\alpha$, $\beta$)\par
 \-Map\-A\-Conjunto(n) $\equiv$ {\bfseries if} n.\-color = \-Header {\bfseries then} \-Map\-A\-Conjunto(n.\-parent) {\bfseries else} \{n.\-value\} $\cup$ {\bfseries if} n.\-child\mbox{[}0\mbox{]} $\_linebr eq$ $\bot$ {\bfseries then} \-Map\-A\-Conjunto($\ast$n.child\mbox{[}0\mbox{]}) {\bfseries fi} $\cup$ {\bfseries if} n.\-child\mbox{[}1\mbox{]} $\_linebr eq$ $\bot$ {\bfseries then} \-Map\-A\-Conjunto($\ast$n.child\mbox{[}1\mbox{]}) {\bfseries fi} {\bfseries fi} 

\begin{DoxyParagraph}{minimo}
\-Devuelve el elemento del diccionario con la menor clave \par
 minimo\-: \-Conj( $\alpha$, $\beta$) $\to$ ( $\alpha$, $\beta$) \{ $\lnot$ $\emptyset$?(elems)\}\par
 minimo(elems) $\equiv$ minimo\-Aux(elems, $\pi_1$(dame\-Uno(elems))) 
\end{DoxyParagraph}
\begin{DoxyParagraph}{minimo\-Aux}
minimo\-Aux\-: \-Conj( $\alpha$, $\beta$) x ( $\alpha$, $\beta$) $\to$ ( $\alpha$, $\beta$)\par
 minimo\-Aux(elems, e) $\equiv$ {\bfseries if} \#elems = 0 {\bfseries then} e {\bfseries else} {\bfseries if} e $>$ $\pi_1$(dame\-Uno(elems)) {\bfseries then} minimo\-Aux(sin\-Uno(elems), e) {\bfseries else} minimo\-Aux(sin\-Uno(elems), dame\-Uno(elems)) {\bfseries fi} 
\end{DoxyParagraph}
\begin{DoxyParagraph}{maximo}
\-Devuelve el elemento del diccionario con la menor clave\par
 maximo\-: \-Conj( $\alpha$, $\beta$) $\to$ ( $\alpha$, $\beta$) \{ $\lnot$ $\emptyset$?(elems)\}\par
 maximo(elems) $\equiv$ maximo\-Aux(elems, $\pi_1$(dame\-Uno(elems))) 
\end{DoxyParagraph}
\begin{DoxyParagraph}{maximo\-Aux}
maximo\-Aux\-: \-Conj( $\alpha$, $\beta$) x ( $\alpha$, $\beta$) $\to$ ( $\alpha$, $\beta$)\par
 maximo\-Aux(elems, e) $\equiv$ {\bfseries if} \#elems = 0 {\bfseries then} e {\bfseries else} {\bfseries if} e $>$ $\pi_1$(dame\-Uno(elems)) {\bfseries then} maximo\-Aux(sin\-Uno(elems), e) {\bfseries else} maximo\-Aux(sin\-Uno(elems), dame\-Uno(elems)) {\bfseries fi}  
\end{DoxyParagraph}
