Como decimos en la materia, que aprendamos a expresar cuestiones en lenguajes formales no invalida que podamos expresarlas (con cierta ambigüedad) en un lenguaje coloquial. Hay muchas ventajas del lenguaje natural, más allá de sus ambigüedades. Es por esto que se espera una descripción en lenguaje coloquial de\+:
\begin{DoxyItemize}
\item todas las especificaciones,
\item todos los algoritmos en términos abstractos, y
\item todas las justificaciones que sean necesarias para entender un concepto (de ejemplo sirve el presente documento),
\end{DoxyItemize}

salvo que las mismas sean triviales.

Asimismo, vamos a permitir describir algunos conceptos usando únicamente un lenguaje coloquial y obviando el lenguaje formal. El objetivo es reducir la carga del trabajo. Esto no necesariamente significa que no se pueda expresar formalmente. En particular, cualquier aspecto que involucre exclusivamente la parte privada del módulo (con excepción de la funciones rep y abs) se puede expresar en castellano. Por ejemplo, la descripción, precondición y postcondición de una función auxiliar puede estar en castellano. Ver los ejemplos que hay en el presente documento. 